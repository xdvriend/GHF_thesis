%% Generated by Sphinx.
\def\sphinxdocclass{report}
\documentclass[letterpaper,10pt,english]{sphinxmanual}
\ifdefined\pdfpxdimen
   \let\sphinxpxdimen\pdfpxdimen\else\newdimen\sphinxpxdimen
\fi \sphinxpxdimen=.75bp\relax

\PassOptionsToPackage{warn}{textcomp}
\usepackage[utf8]{inputenc}
\ifdefined\DeclareUnicodeCharacter
% support both utf8 and utf8x syntaxes
  \ifdefined\DeclareUnicodeCharacterAsOptional
    \def\sphinxDUC#1{\DeclareUnicodeCharacter{"#1}}
  \else
    \let\sphinxDUC\DeclareUnicodeCharacter
  \fi
  \sphinxDUC{00A0}{\nobreakspace}
  \sphinxDUC{2500}{\sphinxunichar{2500}}
  \sphinxDUC{2502}{\sphinxunichar{2502}}
  \sphinxDUC{2514}{\sphinxunichar{2514}}
  \sphinxDUC{251C}{\sphinxunichar{251C}}
  \sphinxDUC{2572}{\textbackslash}
\fi
\usepackage{cmap}
\usepackage[T1]{fontenc}
\usepackage{amsmath,amssymb,amstext}
\usepackage{babel}



\usepackage{times}
\expandafter\ifx\csname T@LGR\endcsname\relax
\else
% LGR was declared as font encoding
  \substitutefont{LGR}{\rmdefault}{cmr}
  \substitutefont{LGR}{\sfdefault}{cmss}
  \substitutefont{LGR}{\ttdefault}{cmtt}
\fi
\expandafter\ifx\csname T@X2\endcsname\relax
  \expandafter\ifx\csname T@T2A\endcsname\relax
  \else
  % T2A was declared as font encoding
    \substitutefont{T2A}{\rmdefault}{cmr}
    \substitutefont{T2A}{\sfdefault}{cmss}
    \substitutefont{T2A}{\ttdefault}{cmtt}
  \fi
\else
% X2 was declared as font encoding
  \substitutefont{X2}{\rmdefault}{cmr}
  \substitutefont{X2}{\sfdefault}{cmss}
  \substitutefont{X2}{\ttdefault}{cmtt}
\fi


\usepackage[Bjarne]{fncychap}
\usepackage{sphinx}

\fvset{fontsize=\small}
\usepackage{geometry}

% Include hyperref last.
\usepackage{hyperref}
% Fix anchor placement for figures with captions.
\usepackage{hypcap}% it must be loaded after hyperref.
% Set up styles of URL: it should be placed after hyperref.
\urlstyle{same}
\addto\captionsenglish{\renewcommand{\contentsname}{Contents:}}

\usepackage{sphinxmessages}
\setcounter{tocdepth}{1}



\title{GHF}
\date{Nov 11, 2019}
\release{0.1}
\author{Xeno De Vriendt}
\newcommand{\sphinxlogo}{\vbox{}}
\renewcommand{\releasename}{Release}
\makeindex
\begin{document}

\pagestyle{empty}
\sphinxmaketitle
\pagestyle{plain}
\sphinxtableofcontents
\pagestyle{normal}
\phantomsection\label{\detokenize{index::doc}}

\phantomsection\label{\detokenize{RHF:module-ghf.RHF}}\index{ghf.RHF (module)@\spxentry{ghf.RHF}\spxextra{module}}

\chapter{Restricted Hartree Fock, by means of SCF procedure}
\label{\detokenize{RHF:restricted-hartree-fock-by-means-of-scf-procedure}}\label{\detokenize{RHF::doc}}
This class is used to calculate the RHF energy of a given molecule and the number of electrons.
The molecule has to be created in pySCF:
molecule = gto.M(atom = geometry, spin = diff. in alpha and beta electrons, basis = basis set)
\index{RHF (class in ghf.RHF)@\spxentry{RHF}\spxextra{class in ghf.RHF}}

\begin{fulllineitems}
\phantomsection\label{\detokenize{RHF:ghf.RHF.RHF}}\pysiglinewithargsret{\sphinxbfcode{\sphinxupquote{class }}\sphinxcode{\sphinxupquote{ghf.RHF.}}\sphinxbfcode{\sphinxupquote{RHF}}}{\emph{molecule}, \emph{number\_of\_electrons}}{}
Input is a molecule and the number of electrons.

Molecules are made in pySCF and calculations are performed as follows, eg.:
The following snippet prints and returns RHF energy of h\_2
and the number of iterations needed to get this value.

\begin{sphinxVerbatim}[commandchars=\\\{\}]
\PYG{g+gp}{\PYGZgt{}\PYGZgt{}\PYGZgt{} }\PYG{n}{h\PYGZus{}2} \PYG{o}{=} \PYG{n}{gto}\PYG{o}{.}\PYG{n}{M}\PYG{p}{(}\PYG{n}{atom} \PYG{o}{=} \PYG{l+s+s1}{\PYGZsq{}}\PYG{l+s+s1}{h 0 0 0; h 0 0 1}\PYG{l+s+s1}{\PYGZsq{}}\PYG{p}{,} \PYG{n}{spin} \PYG{o}{=} \PYG{l+m+mi}{0}\PYG{p}{,} \PYG{n}{basis} \PYG{o}{=} \PYG{l+s+s1}{\PYGZsq{}}\PYG{l+s+s1}{cc\PYGZhy{}pvdz}\PYG{l+s+s1}{\PYGZsq{}}\PYG{p}{)}
\PYG{g+gp}{\PYGZgt{}\PYGZgt{}\PYGZgt{} }\PYG{n}{x} \PYG{o}{=} \PYG{n}{RHF}\PYG{p}{(}\PYG{n}{h\PYGZus{}2}\PYG{p}{,} \PYG{l+m+mi}{2}\PYG{p}{)}
\PYG{g+gp}{\PYGZgt{}\PYGZgt{}\PYGZgt{} }\PYG{n}{x}\PYG{o}{.}\PYG{n}{get\PYGZus{}scf\PYGZus{}solution}\PYG{p}{(}\PYG{p}{)}
\PYG{g+go}{Number of iterations: 109}
\PYG{g+go}{Converged SCF energy in Hartree: \PYGZhy{}1.9403598392831243 (RHF)}
\end{sphinxVerbatim}
\index{get\_last\_dens() (ghf.RHF.RHF method)@\spxentry{get\_last\_dens()}\spxextra{ghf.RHF.RHF method}}

\begin{fulllineitems}
\phantomsection\label{\detokenize{RHF:ghf.RHF.RHF.get_last_dens}}\pysiglinewithargsret{\sphinxbfcode{\sphinxupquote{get\_last\_dens}}}{}{}
Returns the last density matrix of the converged solution.
\begin{quote}\begin{description}
\item[{Returns}] \leavevmode
The last density matrix.

\end{description}\end{quote}

\end{fulllineitems}

\index{get\_last\_fock() (ghf.RHF.RHF method)@\spxentry{get\_last\_fock()}\spxextra{ghf.RHF.RHF method}}

\begin{fulllineitems}
\phantomsection\label{\detokenize{RHF:ghf.RHF.RHF.get_last_fock}}\pysiglinewithargsret{\sphinxbfcode{\sphinxupquote{get\_last\_fock}}}{}{}
Returns the last fock matrix of the converged solution.
\begin{quote}\begin{description}
\item[{Returns}] \leavevmode
The last Fock matrix.

\end{description}\end{quote}

\end{fulllineitems}

\index{get\_mo\_coeff() (ghf.RHF.RHF method)@\spxentry{get\_mo\_coeff()}\spxextra{ghf.RHF.RHF method}}

\begin{fulllineitems}
\phantomsection\label{\detokenize{RHF:ghf.RHF.RHF.get_mo_coeff}}\pysiglinewithargsret{\sphinxbfcode{\sphinxupquote{get\_mo\_coeff}}}{}{}
Returns mo coefficients of the converged solution.
\begin{quote}\begin{description}
\item[{Returns}] \leavevmode
The mo coefficients

\end{description}\end{quote}

\end{fulllineitems}

\index{get\_one\_e() (ghf.RHF.RHF method)@\spxentry{get\_one\_e()}\spxextra{ghf.RHF.RHF method}}

\begin{fulllineitems}
\phantomsection\label{\detokenize{RHF:ghf.RHF.RHF.get_one_e}}\pysiglinewithargsret{\sphinxbfcode{\sphinxupquote{get\_one\_e}}}{}{}~\begin{quote}\begin{description}
\item[{Returns}] \leavevmode
The one electron integral matrix: T + V

\end{description}\end{quote}

\end{fulllineitems}

\index{get\_ovlp() (ghf.RHF.RHF method)@\spxentry{get\_ovlp()}\spxextra{ghf.RHF.RHF method}}

\begin{fulllineitems}
\phantomsection\label{\detokenize{RHF:ghf.RHF.RHF.get_ovlp}}\pysiglinewithargsret{\sphinxbfcode{\sphinxupquote{get\_ovlp}}}{}{}~\begin{quote}\begin{description}
\item[{Returns}] \leavevmode
The overlap matrix

\end{description}\end{quote}

\end{fulllineitems}

\index{get\_scf\_solution() (ghf.RHF.RHF method)@\spxentry{get\_scf\_solution()}\spxextra{ghf.RHF.RHF method}}

\begin{fulllineitems}
\phantomsection\label{\detokenize{RHF:ghf.RHF.RHF.get_scf_solution}}\pysiglinewithargsret{\sphinxbfcode{\sphinxupquote{get\_scf\_solution}}}{}{}
Prints the number of iterations and the converged scf energy.
\begin{quote}\begin{description}
\item[{Returns}] \leavevmode
The converged scf energy.

\end{description}\end{quote}

\end{fulllineitems}

\index{get\_two\_e() (ghf.RHF.RHF method)@\spxentry{get\_two\_e()}\spxextra{ghf.RHF.RHF method}}

\begin{fulllineitems}
\phantomsection\label{\detokenize{RHF:ghf.RHF.RHF.get_two_e}}\pysiglinewithargsret{\sphinxbfcode{\sphinxupquote{get\_two\_e}}}{}{}~\begin{quote}\begin{description}
\item[{Returns}] \leavevmode
The electron repulsion interaction tensor

\end{description}\end{quote}

\end{fulllineitems}

\index{nuc\_rep() (ghf.RHF.RHF method)@\spxentry{nuc\_rep()}\spxextra{ghf.RHF.RHF method}}

\begin{fulllineitems}
\phantomsection\label{\detokenize{RHF:ghf.RHF.RHF.nuc_rep}}\pysiglinewithargsret{\sphinxbfcode{\sphinxupquote{nuc\_rep}}}{}{}~\begin{quote}\begin{description}
\item[{Returns}] \leavevmode
The nuclear repulsion value

\end{description}\end{quote}

\end{fulllineitems}

\index{scf() (ghf.RHF.RHF method)@\spxentry{scf()}\spxextra{ghf.RHF.RHF method}}

\begin{fulllineitems}
\phantomsection\label{\detokenize{RHF:ghf.RHF.RHF.scf}}\pysiglinewithargsret{\sphinxbfcode{\sphinxupquote{scf}}}{}{}
Performs a self consistent field calculation to find the lowest RHF energy.
\begin{quote}\begin{description}
\item[{Returns}] \leavevmode
number of iterations, scf energy, mo coefficients, last density matrix, last fock matrix

\end{description}\end{quote}

\end{fulllineitems}


\end{fulllineitems}

\phantomsection\label{\detokenize{UHF:module-ghf.UHF}}\index{ghf.UHF (module)@\spxentry{ghf.UHF}\spxextra{module}}

\chapter{Unrestricted Hartree Fock, by means of SCF procedure}
\label{\detokenize{UHF:unrestricted-hartree-fock-by-means-of-scf-procedure}}\label{\detokenize{UHF::doc}}
This class is used to calculate the UHF energy for a given molecule and the number of electrons of that molecule.
Several options are available to make sure you get the lowest energy from your calculation, as well as some usefull
functions to get intermediate values such as MO coefficients, density and fock matrices.
\index{UHF (class in ghf.UHF)@\spxentry{UHF}\spxextra{class in ghf.UHF}}

\begin{fulllineitems}
\phantomsection\label{\detokenize{UHF:ghf.UHF.UHF}}\pysiglinewithargsret{\sphinxbfcode{\sphinxupquote{class }}\sphinxcode{\sphinxupquote{ghf.UHF.}}\sphinxbfcode{\sphinxupquote{UHF}}}{\emph{molecule}, \emph{number\_of\_electrons}}{}
Input is a molecule and the number of electrons.

Molecules are made in pySCF and calculations are performed as follows, eg.:
The following snippet prints and returns UHF energy of h\_3
and the number of iterations needed to get this value.

For a normal scf calculation your input looks like the following example:

\begin{sphinxVerbatim}[commandchars=\\\{\}]
\PYG{g+gp}{\PYGZgt{}\PYGZgt{}\PYGZgt{} }\PYG{n}{h3} \PYG{o}{=} \PYG{n}{gto}\PYG{o}{.}\PYG{n}{M}\PYG{p}{(}\PYG{n}{atom} \PYG{o}{=} \PYG{l+s+s1}{\PYGZsq{}}\PYG{l+s+s1}{h 0 0 0; h 0 0.86602540378 0.5; h 0 0 1}\PYG{l+s+s1}{\PYGZsq{}}\PYG{p}{,} \PYG{n}{spin} \PYG{o}{=} \PYG{l+m+mi}{1}\PYG{p}{,} \PYG{n}{basis} \PYG{o}{=} \PYG{l+s+s1}{\PYGZsq{}}\PYG{l+s+s1}{cc\PYGZhy{}pvdz}\PYG{l+s+s1}{\PYGZsq{}}\PYG{p}{)}
\PYG{g+gp}{\PYGZgt{}\PYGZgt{}\PYGZgt{} }\PYG{n}{x} \PYG{o}{=} \PYG{n}{UHF}\PYG{p}{(}\PYG{n}{h3}\PYG{p}{,} \PYG{l+m+mi}{3}\PYG{p}{)}
\PYG{g+gp}{\PYGZgt{}\PYGZgt{}\PYGZgt{} }\PYG{n}{x}\PYG{o}{.}\PYG{n}{get\PYGZus{}scf\PYGZus{}solution}\PYG{p}{(}\PYG{p}{)}
\PYG{g+go}{Number of iterations: 62}
\PYG{g+go}{Converged SCF energy in Hartree: \PYGZhy{}1.5062743202681235 (UHF)}
\PYG{g+go}{\PYGZlt{}S\PYGZca{}2\PYGZgt{} = 0.7735672504295973, \PYGZlt{}S\PYGZus{}z\PYGZgt{} = 0.5, Multiplicity = 2.023430009098014}
\end{sphinxVerbatim}
\index{extra\_electron\_guess() (ghf.UHF.UHF method)@\spxentry{extra\_electron\_guess()}\spxextra{ghf.UHF.UHF method}}

\begin{fulllineitems}
\phantomsection\label{\detokenize{UHF:ghf.UHF.UHF.extra_electron_guess}}\pysiglinewithargsret{\sphinxbfcode{\sphinxupquote{extra\_electron\_guess}}}{}{}
This method adds two electrons to the system in order to get coefficients that can be used as a better guess
for the scf procedure. This essentially forces the system into it’s \textless{}S\_z\textgreater{} = 0 state.

To perform a calculation with this method, you will have to work as follows:

\begin{sphinxVerbatim}[commandchars=\\\{\}]
\PYG{g+gp}{\PYGZgt{}\PYGZgt{}\PYGZgt{} }\PYG{n}{h4} \PYG{o}{=} \PYG{n}{gto}\PYG{o}{.}\PYG{n}{M}\PYG{p}{(}\PYG{n}{atom} \PYG{o}{=} \PYG{l+s+s1}{\PYGZsq{}}\PYG{l+s+s1}{h 0 0 0; h 1 0 0; h 0 1 0; h 1 1 0}\PYG{l+s+s1}{\PYGZsq{}} \PYG{p}{,} \PYG{n}{spin} \PYG{o}{=} \PYG{l+m+mi}{2}\PYG{p}{,} \PYG{n}{basis} \PYG{o}{=} \PYG{l+s+s1}{\PYGZsq{}}\PYG{l+s+s1}{cc\PYGZhy{}pvdz}\PYG{l+s+s1}{\PYGZsq{}}\PYG{p}{)}
\PYG{g+gp}{\PYGZgt{}\PYGZgt{}\PYGZgt{} }\PYG{n}{x} \PYG{o}{=} \PYG{n}{UHF}\PYG{p}{(}\PYG{n}{h4}\PYG{p}{,} \PYG{l+m+mi}{4}\PYG{p}{)}
\PYG{g+gp}{\PYGZgt{}\PYGZgt{}\PYGZgt{} }\PYG{n}{guess} \PYG{o}{=} \PYG{n}{x}\PYG{o}{.}\PYG{n}{extra\PYGZus{}electron\PYGZus{}guess}\PYG{p}{(}\PYG{p}{)}
\PYG{g+gp}{\PYGZgt{}\PYGZgt{}\PYGZgt{} }\PYG{n}{x}\PYG{o}{.}\PYG{n}{get\PYGZus{}scf\PYGZus{}solution}\PYG{p}{(}\PYG{n}{guess}\PYG{p}{)}
\PYG{g+go}{Number of iterations: 74}
\PYG{g+go}{Converged SCF energy in Hartree: \PYGZhy{}2.0210882477030547 (UHF)}
\PYG{g+go}{\PYGZlt{}S\PYGZca{}2\PYGZgt{} = 1.0565277001056579, \PYGZlt{}S\PYGZus{}z\PYGZgt{} = 0.0, Multiplicity = 2.2860688529487976}
\end{sphinxVerbatim}
\begin{quote}\begin{description}
\item[{Returns}] \leavevmode
A new guess matrix to use for the scf procedure.

\end{description}\end{quote}

\end{fulllineitems}

\index{get\_last\_dens() (ghf.UHF.UHF method)@\spxentry{get\_last\_dens()}\spxextra{ghf.UHF.UHF method}}

\begin{fulllineitems}
\phantomsection\label{\detokenize{UHF:ghf.UHF.UHF.get_last_dens}}\pysiglinewithargsret{\sphinxbfcode{\sphinxupquote{get\_last\_dens}}}{}{}
Gets the last density matrix of the converged solution.
Alpha density in the first matrix, beta density in the second.
\begin{quote}\begin{description}
\item[{Returns}] \leavevmode
The last density matrix.

\end{description}\end{quote}

\end{fulllineitems}

\index{get\_last\_fock() (ghf.UHF.UHF method)@\spxentry{get\_last\_fock()}\spxextra{ghf.UHF.UHF method}}

\begin{fulllineitems}
\phantomsection\label{\detokenize{UHF:ghf.UHF.UHF.get_last_fock}}\pysiglinewithargsret{\sphinxbfcode{\sphinxupquote{get\_last\_fock}}}{}{}
Gets the last fock matrix of the converged solution.
Alpha Fock matrix first, beta Fock matrix second.
\begin{quote}\begin{description}
\item[{Returns}] \leavevmode
The last Fock matrix.

\end{description}\end{quote}

\end{fulllineitems}

\index{get\_mo\_coeff() (ghf.UHF.UHF method)@\spxentry{get\_mo\_coeff()}\spxextra{ghf.UHF.UHF method}}

\begin{fulllineitems}
\phantomsection\label{\detokenize{UHF:ghf.UHF.UHF.get_mo_coeff}}\pysiglinewithargsret{\sphinxbfcode{\sphinxupquote{get\_mo\_coeff}}}{}{}
Gets the mo coefficients of the converged solution.
Alpha coefficients in the first matrix, beta coefficients in the second.
\begin{quote}\begin{description}
\item[{Returns}] \leavevmode
The mo coefficients

\end{description}\end{quote}

\end{fulllineitems}

\index{get\_one\_e() (ghf.UHF.UHF method)@\spxentry{get\_one\_e()}\spxextra{ghf.UHF.UHF method}}

\begin{fulllineitems}
\phantomsection\label{\detokenize{UHF:ghf.UHF.UHF.get_one_e}}\pysiglinewithargsret{\sphinxbfcode{\sphinxupquote{get\_one\_e}}}{}{}~\begin{quote}\begin{description}
\item[{Returns}] \leavevmode
The one electron integral matrix: T + V

\end{description}\end{quote}

\end{fulllineitems}

\index{get\_ovlp() (ghf.UHF.UHF method)@\spxentry{get\_ovlp()}\spxextra{ghf.UHF.UHF method}}

\begin{fulllineitems}
\phantomsection\label{\detokenize{UHF:ghf.UHF.UHF.get_ovlp}}\pysiglinewithargsret{\sphinxbfcode{\sphinxupquote{get\_ovlp}}}{}{}~\begin{quote}\begin{description}
\item[{Returns}] \leavevmode
The overlap matrix

\end{description}\end{quote}

\end{fulllineitems}

\index{get\_scf\_solution() (ghf.UHF.UHF method)@\spxentry{get\_scf\_solution()}\spxextra{ghf.UHF.UHF method}}

\begin{fulllineitems}
\phantomsection\label{\detokenize{UHF:ghf.UHF.UHF.get_scf_solution}}\pysiglinewithargsret{\sphinxbfcode{\sphinxupquote{get\_scf\_solution}}}{\emph{guess=None}}{}
Prints the number of iterations and the converged scf energy.
Also prints the expectation value of S\_z, S\textasciicircum{}2 and the multiplicity.
\begin{quote}\begin{description}
\item[{Returns}] \leavevmode
The converged scf energy.

\end{description}\end{quote}

\end{fulllineitems}

\index{get\_two\_e() (ghf.UHF.UHF method)@\spxentry{get\_two\_e()}\spxextra{ghf.UHF.UHF method}}

\begin{fulllineitems}
\phantomsection\label{\detokenize{UHF:ghf.UHF.UHF.get_two_e}}\pysiglinewithargsret{\sphinxbfcode{\sphinxupquote{get\_two\_e}}}{}{}~\begin{quote}\begin{description}
\item[{Returns}] \leavevmode
The electron repulsion interaction tensor

\end{description}\end{quote}

\end{fulllineitems}

\index{nuc\_rep() (ghf.UHF.UHF method)@\spxentry{nuc\_rep()}\spxextra{ghf.UHF.UHF method}}

\begin{fulllineitems}
\phantomsection\label{\detokenize{UHF:ghf.UHF.UHF.nuc_rep}}\pysiglinewithargsret{\sphinxbfcode{\sphinxupquote{nuc\_rep}}}{}{}~\begin{quote}\begin{description}
\item[{Returns}] \leavevmode
The nuclear repulsion value

\end{description}\end{quote}

\end{fulllineitems}

\index{scf() (ghf.UHF.UHF method)@\spxentry{scf()}\spxextra{ghf.UHF.UHF method}}

\begin{fulllineitems}
\phantomsection\label{\detokenize{UHF:ghf.UHF.UHF.scf}}\pysiglinewithargsret{\sphinxbfcode{\sphinxupquote{scf}}}{\emph{initial\_guess=None}}{}
Performs a self consistent field calculation to find the lowest UHF energy.
\begin{quote}\begin{description}
\item[{Parameters}] \leavevmode
\sphinxstyleliteralstrong{\sphinxupquote{initial\_guess}} \textendash{} A tuple of an alpha and beta guess matrix. If none, the core hamiltonian will be used.

\item[{Returns}] \leavevmode
The scf energy, number of iterations, the mo coefficients, the last density and the last fock matrices

\end{description}\end{quote}

\end{fulllineitems}

\index{stability() (ghf.UHF.UHF method)@\spxentry{stability()}\spxextra{ghf.UHF.UHF method}}

\begin{fulllineitems}
\phantomsection\label{\detokenize{UHF:ghf.UHF.UHF.stability}}\pysiglinewithargsret{\sphinxbfcode{\sphinxupquote{stability}}}{}{}
Performing a stability analysis checks whether or not the wave function is stable, by checking the lowest eigen-
value of the Hessian matrix. If there’s an instability, the MO’s will be rotated in the direction
of the lowest eigenvalue. These new MO’s can then be used to start a new scf procedure.

To perform a stability analysis, use the following syntax:

\begin{sphinxVerbatim}[commandchars=\\\{\}]
\PYG{g+gp}{\PYGZgt{}\PYGZgt{}\PYGZgt{} }\PYG{n}{h4} \PYG{o}{=} \PYG{n}{gto}\PYG{o}{.}\PYG{n}{M}\PYG{p}{(}\PYG{n}{atom} \PYG{o}{=} \PYG{l+s+s1}{\PYGZsq{}}\PYG{l+s+s1}{h 0 0 0; h 1 0 0; h 0 1 0; h 1 1 0}\PYG{l+s+s1}{\PYGZsq{}} \PYG{p}{,} \PYG{n}{spin} \PYG{o}{=} \PYG{l+m+mi}{2}\PYG{p}{,} \PYG{n}{basis} \PYG{o}{=} \PYG{l+s+s1}{\PYGZsq{}}\PYG{l+s+s1}{cc\PYGZhy{}pvdz}\PYG{l+s+s1}{\PYGZsq{}}\PYG{p}{)}
\PYG{g+gp}{\PYGZgt{}\PYGZgt{}\PYGZgt{} }\PYG{n}{x} \PYG{o}{=} \PYG{n}{UHF}\PYG{p}{(}\PYG{n}{h4}\PYG{p}{,} \PYG{l+m+mi}{4}\PYG{p}{)}
\PYG{g+gp}{\PYGZgt{}\PYGZgt{}\PYGZgt{} }\PYG{n}{guess} \PYG{o}{=} \PYG{n}{x}\PYG{o}{.}\PYG{n}{stability}\PYG{p}{(}\PYG{p}{)}
\PYG{g+gp}{\PYGZgt{}\PYGZgt{}\PYGZgt{} }\PYG{n}{x}\PYG{o}{.}\PYG{n}{get\PYGZus{}scf\PYGZus{}solution}\PYG{p}{(}\PYG{n}{guess}\PYG{p}{)}
\PYG{g+go}{There is an internal instability in the UHF wave function.}
\PYG{g+go}{Number of iterations: 78}
\PYG{g+go}{Converged SCF energy in Hartree: \PYGZhy{}2.0210882477030716 (UHF)}
\PYG{g+go}{\PYGZlt{}S\PYGZca{}2\PYGZgt{} = 1.056527700105677, \PYGZlt{}S\PYGZus{}z\PYGZgt{} = 0.0, Multiplicity = 2.2860688529488145}
\end{sphinxVerbatim}
\begin{quote}\begin{description}
\item[{Returns}] \leavevmode
New and improved MO’s.

\end{description}\end{quote}

\end{fulllineitems}


\end{fulllineitems}

\phantomsection\label{\detokenize{real_GHF:module-ghf.real_GHF}}\index{ghf.real\_GHF (module)@\spxentry{ghf.real\_GHF}\spxextra{module}}

\chapter{Real generalised Hartree Fock, by means of SCF procedure}
\label{\detokenize{real_GHF:real-generalised-hartree-fock-by-means-of-scf-procedure}}\label{\detokenize{real_GHF::doc}}
This class creates a generalised Hartree-Fock object which can be used for scf calculations. Different initial guesses
are provided as well as the option to perform a stability analysis.
The molecule has to be created in pySCF:
molecule = gto.M(atom = geometry, spin = diff. in alpha and beta electrons, basis = basis set)
\index{RealGHF (class in ghf.real\_GHF)@\spxentry{RealGHF}\spxextra{class in ghf.real\_GHF}}

\begin{fulllineitems}
\phantomsection\label{\detokenize{real_GHF:ghf.real_GHF.RealGHF}}\pysiglinewithargsret{\sphinxbfcode{\sphinxupquote{class }}\sphinxcode{\sphinxupquote{ghf.real\_GHF.}}\sphinxbfcode{\sphinxupquote{RealGHF}}}{\emph{molecule}, \emph{number\_of\_electrons}}{}
Input is a molecule and the number of electrons.

Molecules are made in pySCF and calculations are performed as follows, eg.:
The following snippet prints and returns UHF energy of h3
and the number of iterations needed to get this value.

For a normal scf calculation your input looks like the following example:

\begin{sphinxVerbatim}[commandchars=\\\{\}]
\PYG{g+gp}{\PYGZgt{}\PYGZgt{}\PYGZgt{} }\PYG{n}{h3} \PYG{o}{=} \PYG{n}{gto}\PYG{o}{.}\PYG{n}{M}\PYG{p}{(}\PYG{n}{atom} \PYG{o}{=} \PYG{l+s+s1}{\PYGZsq{}}\PYG{l+s+s1}{h 0 0 0; h 0 0.86602540378 0.5; h 0 0 1}\PYG{l+s+s1}{\PYGZsq{}}\PYG{p}{,} \PYG{n}{spin} \PYG{o}{=} \PYG{l+m+mi}{1}\PYG{p}{,} \PYG{n}{basis} \PYG{o}{=} \PYG{l+s+s1}{\PYGZsq{}}\PYG{l+s+s1}{cc\PYGZhy{}pvdz}\PYG{l+s+s1}{\PYGZsq{}}\PYG{p}{)}
\PYG{g+gp}{\PYGZgt{}\PYGZgt{}\PYGZgt{} }\PYG{n}{x} \PYG{o}{=} \PYG{n}{RealGHF}\PYG{p}{(}\PYG{n}{h3}\PYG{p}{,} \PYG{l+m+mi}{3}\PYG{p}{)}
\PYG{g+gp}{\PYGZgt{}\PYGZgt{}\PYGZgt{} }\PYG{n}{x}\PYG{o}{.} \PYG{n}{get\PYGZus{}scf\PYGZus{}solution}\PYG{p}{(}\PYG{p}{)}
\PYG{g+go}{Number of iterations: 82}
\PYG{g+go}{Converged SCF energy in Hartree: \PYGZhy{}1.5062743202607725 (Real GHF)}
\end{sphinxVerbatim}
\index{get\_last\_dens() (ghf.real\_GHF.RealGHF method)@\spxentry{get\_last\_dens()}\spxextra{ghf.real\_GHF.RealGHF method}}

\begin{fulllineitems}
\phantomsection\label{\detokenize{real_GHF:ghf.real_GHF.RealGHF.get_last_dens}}\pysiglinewithargsret{\sphinxbfcode{\sphinxupquote{get\_last\_dens}}}{}{}
Gets the last density matrix of the converged solution.
\begin{quote}\begin{description}
\item[{Returns}] \leavevmode
The last density matrix.

\end{description}\end{quote}

\end{fulllineitems}

\index{get\_last\_fock() (ghf.real\_GHF.RealGHF method)@\spxentry{get\_last\_fock()}\spxextra{ghf.real\_GHF.RealGHF method}}

\begin{fulllineitems}
\phantomsection\label{\detokenize{real_GHF:ghf.real_GHF.RealGHF.get_last_fock}}\pysiglinewithargsret{\sphinxbfcode{\sphinxupquote{get\_last\_fock}}}{}{}
Gets the last fock matrix of the converged solution.
\begin{quote}\begin{description}
\item[{Returns}] \leavevmode
The last Fock matrix.

\end{description}\end{quote}

\end{fulllineitems}

\index{get\_mo\_coeff() (ghf.real\_GHF.RealGHF method)@\spxentry{get\_mo\_coeff()}\spxextra{ghf.real\_GHF.RealGHF method}}

\begin{fulllineitems}
\phantomsection\label{\detokenize{real_GHF:ghf.real_GHF.RealGHF.get_mo_coeff}}\pysiglinewithargsret{\sphinxbfcode{\sphinxupquote{get\_mo\_coeff}}}{}{}
Gets the mo coefficients of the converged solution.
\begin{quote}\begin{description}
\item[{Returns}] \leavevmode
The mo coefficients

\end{description}\end{quote}

\end{fulllineitems}

\index{get\_one\_e() (ghf.real\_GHF.RealGHF method)@\spxentry{get\_one\_e()}\spxextra{ghf.real\_GHF.RealGHF method}}

\begin{fulllineitems}
\phantomsection\label{\detokenize{real_GHF:ghf.real_GHF.RealGHF.get_one_e}}\pysiglinewithargsret{\sphinxbfcode{\sphinxupquote{get\_one\_e}}}{}{}~\begin{quote}\begin{description}
\item[{Returns}] \leavevmode
The one electron integral matrix: T + V

\end{description}\end{quote}

\end{fulllineitems}

\index{get\_ovlp() (ghf.real\_GHF.RealGHF method)@\spxentry{get\_ovlp()}\spxextra{ghf.real\_GHF.RealGHF method}}

\begin{fulllineitems}
\phantomsection\label{\detokenize{real_GHF:ghf.real_GHF.RealGHF.get_ovlp}}\pysiglinewithargsret{\sphinxbfcode{\sphinxupquote{get\_ovlp}}}{}{}~\begin{quote}\begin{description}
\item[{Returns}] \leavevmode
The overlap matrix

\end{description}\end{quote}

\end{fulllineitems}

\index{get\_scf\_solution() (ghf.real\_GHF.RealGHF method)@\spxentry{get\_scf\_solution()}\spxextra{ghf.real\_GHF.RealGHF method}}

\begin{fulllineitems}
\phantomsection\label{\detokenize{real_GHF:ghf.real_GHF.RealGHF.get_scf_solution}}\pysiglinewithargsret{\sphinxbfcode{\sphinxupquote{get\_scf\_solution}}}{\emph{guess=None}}{}
Prints the number of iterations and the converged scf energy.
\begin{quote}\begin{description}
\item[{Returns}] \leavevmode
The converged scf energy.

\end{description}\end{quote}

\end{fulllineitems}

\index{get\_two\_e() (ghf.real\_GHF.RealGHF method)@\spxentry{get\_two\_e()}\spxextra{ghf.real\_GHF.RealGHF method}}

\begin{fulllineitems}
\phantomsection\label{\detokenize{real_GHF:ghf.real_GHF.RealGHF.get_two_e}}\pysiglinewithargsret{\sphinxbfcode{\sphinxupquote{get\_two\_e}}}{}{}~\begin{quote}\begin{description}
\item[{Returns}] \leavevmode
The electron repulsion interaction tensor

\end{description}\end{quote}

\end{fulllineitems}

\index{nuc\_rep() (ghf.real\_GHF.RealGHF method)@\spxentry{nuc\_rep()}\spxextra{ghf.real\_GHF.RealGHF method}}

\begin{fulllineitems}
\phantomsection\label{\detokenize{real_GHF:ghf.real_GHF.RealGHF.nuc_rep}}\pysiglinewithargsret{\sphinxbfcode{\sphinxupquote{nuc\_rep}}}{}{}~\begin{quote}\begin{description}
\item[{Returns}] \leavevmode
The nuclear repulsion value

\end{description}\end{quote}

\end{fulllineitems}

\index{random\_guess() (ghf.real\_GHF.RealGHF method)@\spxentry{random\_guess()}\spxextra{ghf.real\_GHF.RealGHF method}}

\begin{fulllineitems}
\phantomsection\label{\detokenize{real_GHF:ghf.real_GHF.RealGHF.random_guess}}\pysiglinewithargsret{\sphinxbfcode{\sphinxupquote{random\_guess}}}{}{}
A function that creates a matrix with random values that can be used as an initial guess
for the SCF calculations.

To use this guess:
\textgreater{}\textgreater{}\textgreater{} h3 = gto.M(atom = ‘h 0 0 0; h 0 0.86602540378 0.5; h 0 0 1’, spin = 1, basis = ‘cc-pvdz’)
\textgreater{}\textgreater{}\textgreater{} x = RealGHF(h3, 3)
\textgreater{}\textgreater{}\textgreater{} guess = x.random\_guess()
\textgreater{}\textgreater{}\textgreater{} x.get\_scf\_solution(guess)
:return: A random hermitian matrix.

\end{fulllineitems}

\index{scf() (ghf.real\_GHF.RealGHF method)@\spxentry{scf()}\spxextra{ghf.real\_GHF.RealGHF method}}

\begin{fulllineitems}
\phantomsection\label{\detokenize{real_GHF:ghf.real_GHF.RealGHF.scf}}\pysiglinewithargsret{\sphinxbfcode{\sphinxupquote{scf}}}{\emph{guess=None}}{}
This function performs the SCF calculation by using the generalised Hartree-Fock formulas. Since we’re working
in the real class, all values throughout are real. For complex, see the “complex\_GHF” class.
:param guess: The initial guess to start the calculation. Different options are integrated within the class.
If no guess is specified, the core hamiltonian will be used.
:return: The scf energy, number of iterations, the mo coefficients, the last density and the last fock matrices.

\end{fulllineitems}

\index{stability() (ghf.real\_GHF.RealGHF method)@\spxentry{stability()}\spxextra{ghf.real\_GHF.RealGHF method}}

\begin{fulllineitems}
\phantomsection\label{\detokenize{real_GHF:ghf.real_GHF.RealGHF.stability}}\pysiglinewithargsret{\sphinxbfcode{\sphinxupquote{stability}}}{}{}
Performing a stability analysis checks whether or not the wave function is stable, by checking the lowest eigen-
value of the Hessian matrix. If there’s an instability, the MO’s will be rotated in the direction
of the lowest eigenvalue. These new MO’s can then be used to start a new scf procedure.

To perform a stability analysis, use the following syntax, this will continue the analysis until there is
no more instability:
\textgreater{}\textgreater{}\textgreater{} h4 = gto.M(atom = ‘h 0 0 0; h 1 0 0; h 0 1 0; h 1 1 0’ , spin = 2, basis = ‘cc-pvdz’)
\textgreater{}\textgreater{}\textgreater{} x = RealGHF(h4, 4)
\textgreater{}\textgreater{}\textgreater{} x.scf()
\textgreater{}\textgreater{}\textgreater{} guess = x.stability()
\textgreater{}\textgreater{}\textgreater{} while x.instability:
\textgreater{}\textgreater{}\textgreater{}     new\_guess = x.stability()
\textgreater{}\textgreater{}\textgreater{}     x.get\_scf\_solution(new\_guess)
\begin{quote}\begin{description}
\item[{Returns}] \leavevmode
New and improved MO’s.

\end{description}\end{quote}

\end{fulllineitems}

\index{unitary\_rotation\_guess() (ghf.real\_GHF.RealGHF method)@\spxentry{unitary\_rotation\_guess()}\spxextra{ghf.real\_GHF.RealGHF method}}

\begin{fulllineitems}
\phantomsection\label{\detokenize{real_GHF:ghf.real_GHF.RealGHF.unitary_rotation_guess}}\pysiglinewithargsret{\sphinxbfcode{\sphinxupquote{unitary\_rotation\_guess}}}{}{}
A function that creates an initial guess matrix by performing a unitary transformation on the core Hamiltonian
matrix.

To use this guess:
\textgreater{}\textgreater{}\textgreater{} h3 = gto.M(atom = ‘h 0 0 0; h 0 0.86602540378 0.5; h 0 0 1’, spin = 1, basis = ‘cc-pvdz’)
\textgreater{}\textgreater{}\textgreater{} x = RealGHF(h3, 3)
\textgreater{}\textgreater{}\textgreater{} guess = x.unitary\_rotation\_guess()
\textgreater{}\textgreater{}\textgreater{} x.get\_scf\_solution(guess)
:return: A rotated guess matrix.

\end{fulllineitems}


\end{fulllineitems}

\phantomsection\label{\detokenize{SCF_functions:module-ghf.SCF_functions}}\index{ghf.SCF\_functions (module)@\spxentry{ghf.SCF\_functions}\spxextra{module}}

\chapter{Useful functions for SCF procedure}
\label{\detokenize{SCF_functions:useful-functions-for-scf-procedure}}\label{\detokenize{SCF_functions::doc}}
A number of functions used throughout the UHF and RHF calculations are summarised here.
\index{density\_matrix() (in module ghf.SCF\_functions)@\spxentry{density\_matrix()}\spxextra{in module ghf.SCF\_functions}}

\begin{fulllineitems}
\phantomsection\label{\detokenize{SCF_functions:ghf.SCF_functions.density_matrix}}\pysiglinewithargsret{\sphinxcode{\sphinxupquote{ghf.SCF\_functions.}}\sphinxbfcode{\sphinxupquote{density\_matrix}}}{\emph{f\_matrix}, \emph{occ}, \emph{trans}}{}~\begin{itemize}
\item {} 
density() creates a density matrix from a fock matrix and the number of occupied orbitals.

\item {} 
Input is a fock matrix, the number of occupied orbitals, which can be separate for alpha and beta in case of UHF.
And a transformation matrix X.

\end{itemize}

\end{fulllineitems}

\index{expand\_matrix() (in module ghf.SCF\_functions)@\spxentry{expand\_matrix()}\spxextra{in module ghf.SCF\_functions}}

\begin{fulllineitems}
\phantomsection\label{\detokenize{SCF_functions:ghf.SCF_functions.expand_matrix}}\pysiglinewithargsret{\sphinxcode{\sphinxupquote{ghf.SCF\_functions.}}\sphinxbfcode{\sphinxupquote{expand\_matrix}}}{\emph{matrix}}{}~\begin{quote}\begin{description}
\item[{Parameters}] \leavevmode
\sphinxstyleliteralstrong{\sphinxupquote{matrix}} \textendash{} 

\item[{Returns}] \leavevmode
a matrix double the size, where blocks of zero’s are added top right and bottom left.

\end{description}\end{quote}

\end{fulllineitems}

\index{get\_integrals() (in module ghf.SCF\_functions)@\spxentry{get\_integrals()}\spxextra{in module ghf.SCF\_functions}}

\begin{fulllineitems}
\phantomsection\label{\detokenize{SCF_functions:ghf.SCF_functions.get_integrals}}\pysiglinewithargsret{\sphinxcode{\sphinxupquote{ghf.SCF\_functions.}}\sphinxbfcode{\sphinxupquote{get\_integrals}}}{\emph{molecule}}{}
A function to calculate your integrals \& nuclear repulsion with pyscf.

\end{fulllineitems}

\index{spin() (in module ghf.SCF\_functions)@\spxentry{spin()}\spxextra{in module ghf.SCF\_functions}}

\begin{fulllineitems}
\phantomsection\label{\detokenize{SCF_functions:ghf.SCF_functions.spin}}\pysiglinewithargsret{\sphinxcode{\sphinxupquote{ghf.SCF\_functions.}}\sphinxbfcode{\sphinxupquote{spin}}}{\emph{occ\_a}, \emph{occ\_b}, \emph{coeff\_a}, \emph{coeff\_b}, \emph{overlap}}{}~\begin{quote}\begin{description}
\item[{Parameters}] \leavevmode\begin{itemize}
\item {} 
\sphinxstyleliteralstrong{\sphinxupquote{occ\_a}} \textendash{} number of occupied alpha orbitals

\item {} 
\sphinxstyleliteralstrong{\sphinxupquote{occ\_b}} \textendash{} number of occupied beta orbitals

\item {} 
\sphinxstyleliteralstrong{\sphinxupquote{coeff\_a}} \textendash{} MO coefficients of alpha orbitals

\item {} 
\sphinxstyleliteralstrong{\sphinxupquote{coeff\_b}} \textendash{} MO coefficients of beta orbitals

\item {} 
\sphinxstyleliteralstrong{\sphinxupquote{overlap}} \textendash{} overlap matrix of the molecule

\end{itemize}

\item[{Returns}] \leavevmode
S\textasciicircum{}2, S\_z and spin multiplicity

\end{description}\end{quote}

\end{fulllineitems}

\index{spin\_blocked() (in module ghf.SCF\_functions)@\spxentry{spin\_blocked()}\spxextra{in module ghf.SCF\_functions}}

\begin{fulllineitems}
\phantomsection\label{\detokenize{SCF_functions:ghf.SCF_functions.spin_blocked}}\pysiglinewithargsret{\sphinxcode{\sphinxupquote{ghf.SCF\_functions.}}\sphinxbfcode{\sphinxupquote{spin\_blocked}}}{\emph{block\_1}, \emph{block\_2}, \emph{block\_3}, \emph{block\_4}}{}
When creating the blocks of the density or fock matrix separately, this function is used to add them together,
and create the total density or Fock matrix in spin Blocked notation.
:return: a density matrix in the spin-blocked notation

\end{fulllineitems}

\index{trans\_matrix() (in module ghf.SCF\_functions)@\spxentry{trans\_matrix()}\spxextra{in module ghf.SCF\_functions}}

\begin{fulllineitems}
\phantomsection\label{\detokenize{SCF_functions:ghf.SCF_functions.trans_matrix}}\pysiglinewithargsret{\sphinxcode{\sphinxupquote{ghf.SCF\_functions.}}\sphinxbfcode{\sphinxupquote{trans\_matrix}}}{\emph{overlap}}{}~\begin{itemize}
\item {} 
Define a transformation matrix X, used to orthogonalize different matrices throughout the calculation.

\item {} 
Input should be an overlap matrix.

\end{itemize}

\end{fulllineitems}

\index{uhf\_fock\_matrix() (in module ghf.SCF\_functions)@\spxentry{uhf\_fock\_matrix()}\spxextra{in module ghf.SCF\_functions}}

\begin{fulllineitems}
\phantomsection\label{\detokenize{SCF_functions:ghf.SCF_functions.uhf_fock_matrix}}\pysiglinewithargsret{\sphinxcode{\sphinxupquote{ghf.SCF\_functions.}}\sphinxbfcode{\sphinxupquote{uhf\_fock\_matrix}}}{\emph{density\_matrix\_1}, \emph{density\_matrix\_2}, \emph{one\_electron}, \emph{two\_electron}}{}~\begin{itemize}
\item {} 
calculate a fock matrix from a given alpha and beta density matrix

\item {} 
fock alpha if 1 = alpha and 2 = beta and vice versa

\item {} 
input is the density matrix for alpha and beta, a one electron matrix and a two electron tensor.

\end{itemize}

\end{fulllineitems}

\index{uhf\_scf\_energy() (in module ghf.SCF\_functions)@\spxentry{uhf\_scf\_energy()}\spxextra{in module ghf.SCF\_functions}}

\begin{fulllineitems}
\phantomsection\label{\detokenize{SCF_functions:ghf.SCF_functions.uhf_scf_energy}}\pysiglinewithargsret{\sphinxcode{\sphinxupquote{ghf.SCF\_functions.}}\sphinxbfcode{\sphinxupquote{uhf\_scf\_energy}}}{\emph{density\_matrix\_a}, \emph{density\_matrix\_b}, \emph{fock\_a}, \emph{fock\_b}, \emph{one\_electron}}{}~\begin{itemize}
\item {} 
calculate the scf energy value from a given density matrix and a given fock matrix for both alpha and beta,
so 4 matrices in total.

\item {} 
then calculate the initial electronic energy and put it into an array

\item {} 
input is the density matrices for alpha and beta, the fock matrices for alpha and beta and lastly a one electron
matrix.

\end{itemize}

\end{fulllineitems}

\phantomsection\label{\detokenize{tests:module-ghf.tests.test_auth}}\index{ghf.tests.test\_auth (module)@\spxentry{ghf.tests.test\_auth}\spxextra{module}}

\chapter{Testing the RHF and UHF methods}
\label{\detokenize{tests:testing-the-rhf-and-uhf-methods}}\label{\detokenize{tests::doc}}
Simple tests to check whether or not the functions return the correct value.
\index{test\_RHF() (in module ghf.tests.test\_auth)@\spxentry{test\_RHF()}\spxextra{in module ghf.tests.test\_auth}}

\begin{fulllineitems}
\phantomsection\label{\detokenize{tests:ghf.tests.test_auth.test_RHF}}\pysiglinewithargsret{\sphinxcode{\sphinxupquote{ghf.tests.test\_auth.}}\sphinxbfcode{\sphinxupquote{test\_RHF}}}{}{}
test\_RHF will test whether or not the RHF method returns the wanted result. The accuracy is 10\textasciicircum{}11.

\end{fulllineitems}

\index{test\_UHF() (in module ghf.tests.test\_auth)@\spxentry{test\_UHF()}\spxextra{in module ghf.tests.test\_auth}}

\begin{fulllineitems}
\phantomsection\label{\detokenize{tests:ghf.tests.test_auth.test_UHF}}\pysiglinewithargsret{\sphinxcode{\sphinxupquote{ghf.tests.test\_auth.}}\sphinxbfcode{\sphinxupquote{test\_UHF}}}{}{}
test\_UHF will test the regular UHF method, by checking whether or not it returns the expected result. The accuracy is 10\textasciicircum{}-6.

\end{fulllineitems}

\index{test\_extra\_e() (in module ghf.tests.test\_auth)@\spxentry{test\_extra\_e()}\spxextra{in module ghf.tests.test\_auth}}

\begin{fulllineitems}
\phantomsection\label{\detokenize{tests:ghf.tests.test_auth.test_extra_e}}\pysiglinewithargsret{\sphinxcode{\sphinxupquote{ghf.tests.test\_auth.}}\sphinxbfcode{\sphinxupquote{test\_extra\_e}}}{}{}
test\_extra\_e will test the UHF method, with the added option of first adding 2 electrons to the system and using those coefficients
for the actual system, by checking whether or not it returns the expected result. The accuracy is 10\textasciicircum{}-6.

\end{fulllineitems}

\index{test\_stability() (in module ghf.tests.test\_auth)@\spxentry{test\_stability()}\spxextra{in module ghf.tests.test\_auth}}

\begin{fulllineitems}
\phantomsection\label{\detokenize{tests:ghf.tests.test_auth.test_stability}}\pysiglinewithargsret{\sphinxcode{\sphinxupquote{ghf.tests.test\_auth.}}\sphinxbfcode{\sphinxupquote{test\_stability}}}{}{}
test\_stability will test the UHF method, with stability analysis, by checking whether or not it returns the expected result. The accuracy is 10\textasciicircum{}-6.

\end{fulllineitems}



\renewcommand{\indexname}{Python Module Index}
\begin{sphinxtheindex}
\let\bigletter\sphinxstyleindexlettergroup
\bigletter{g}
\item\relax\sphinxstyleindexentry{ghf.real\_GHF}\sphinxstyleindexpageref{real_GHF:\detokenize{module-ghf.real_GHF}}
\item\relax\sphinxstyleindexentry{ghf.RHF}\sphinxstyleindexpageref{RHF:\detokenize{module-ghf.RHF}}
\item\relax\sphinxstyleindexentry{ghf.SCF\_functions}\sphinxstyleindexpageref{SCF_functions:\detokenize{module-ghf.SCF_functions}}
\item\relax\sphinxstyleindexentry{ghf.tests.test\_auth}\sphinxstyleindexpageref{tests:\detokenize{module-ghf.tests.test_auth}}
\item\relax\sphinxstyleindexentry{ghf.UHF}\sphinxstyleindexpageref{UHF:\detokenize{module-ghf.UHF}}
\end{sphinxtheindex}

\renewcommand{\indexname}{Index}
\printindex
\end{document}